\documentclass[a4paper]{article}
\usepackage[utf8]{inputenc}
\usepackage[slovene]{babel}
\usepackage{graphicx}
\usepackage{hyperref}
\usepackage[nottoc]{tocbibind}
\usepackage{minted}
\usepackage{listings}
\usepackage{caption}
\usepackage{subcaption}
\usepackage{amsmath}
\usepackage{ dsfont }
\usepackage{siunitx}
\usepackage{multimedia}
\usepackage[table,xcdraw]{xcolor}
\setlength\parindent{0pt}

\definecolor{codegreen}{rgb}{0,0.6,0}
\definecolor{codegray}{rgb}{0.5,0.5,0.5}
\definecolor{codepurple}{rgb}{0.58,0,0.82}
\definecolor{backcolour}{rgb}{0.95,0.95,0.92}
\newcommand{\ddd}{\mathrm{d}}
\newcommand\myworries[1]{\textcolor{red}{#1}}
\newcommand{\Dd}[3][{}]{\frac{\ddd^{#1} #2}{\ddd #3^{#1}}}

\lstdefinestyle{mystyle}{
    backgroundcolor=\color{backcolour},   
    commentstyle=\color{codegreen},
    keywordstyle=\color{magenta},
    numberstyle=\tiny\color{codegray},
    stringstyle=\color{codepurple},
    basicstyle=\ttfamily\footnotesize,
    breakatwhitespace=false,         
    breaklines=true,                 
    captionpos=b,                    
    keepspaces=true,                 
    numbers=left,                    
    numbersep=5pt,                  
    showspaces=false,                
    showstringspaces=false,
    showtabs=false,                  
    tabsize=2
}

\lstset{style=mystyle}

\begin{document}
\begin{titlepage}
    \begin{center}
        \includegraphics[]{logo.png}
        \vspace*{3cm}
        
        \Huge
        \textbf{Diferenčne metode za parcialne diferencialne enačbe}
        
        \vspace{0.5cm}
        \large
        10. naloga pri Matematično-fizikalnem praktikumu

        \vspace{4.5cm}
        
        \textbf{Avtor:} Marko Urbanč (28191096)\ \\
        \textbf{Predavatelj:} prof. dr. Borut Paul Kerševan\ \\
        
        \vspace{2.8cm}
        
        \large
        4.9.2023
    \end{center}
\end{titlepage}
\tableofcontents
\newpage
\section{Uvod}
V prejšnji nalogi smo rekli, da obstajata v glavnem dva velika razreda za reševanje parcialnih diferencialnih
enačb (PDE). To sta spektralne metode, ki smo jih raziskali v prejšnji nalogi in pa diferencialne metode, ki jih
spoznamo tu. Diferencialne metode so v glavnem metode, ki rešujejo PDE tako, da jih diskretizirajo in jih
pretvorijo v sistem linearnih enačb. Te metode so v glavnem zelo podobne kot metode za reševanje sistemov
navadnih diferencialnih enačb (ODE). V glavnem se razlikujejo v tem, da so PDE lahko tudi nelinearne in
moramo posledično uporabiti iterativne metode za reševanje sistemov linearnih enačb. Tu bomo spoznali 
metodo končnih diferenc (FDM). Ta temleji na Taylorjevem razvoju s katerim lahko aproksimiramo odvod
funkcije. To aproksimacijo nato vstavimo v PDE in dobimo sistem linearnih enačb. Ta sistem nato rešimo
iterativno in dobimo končno rešitev. \\

Fizikalni kontekst za to nalogo bo reševanje enodimenzionalne nestacionarne Schrödingerjeve enačbe, ki se
glasi

\begin{equation}
    \left( i\hbar \frac{\partial}{\partial t} - H\right) \psi(x,\>t) = 0\>.
    \label{schrodinger}
\end{equation}

Predstavlja osnovno orodje za nerelativistični opis kvantnih sistemov. V enačbi \ref{schrodinger} je $H$ Hamiltonian
sistema, ki je v splošnem odvisen od časa. V našem primeru bomo obravnavali časovno neodvisen Hamiltonian

\begin{equation}
    H = -\frac{\hbar^2}{2m} \frac{\partial^2}{\partial x^2} + V(x)\>.
    \label{hamiltonian}
\end{equation}

Z menjavo spremenljivk $H/\hbar \rightarrow H$, $x\sqrt{m/\hbar} \rightarrow x$ efektivno postavimo $\hbar = m = 1$.
V tem primeru je Hamiltonian enak

\begin{equation}
    H = -\frac{1}{2} \frac{\partial^2}{\partial x^2} + V(x)\>.
    \label{hamiltonian2}
\end{equation}

Razvoj stanja $\psi(x,\>t)$ v času $\psi(x,\>t + \Delta t)$ je opisan z približkom

\begin{equation}
    \psi(x,\>t + \Delta t) = e^{-iH\Delta t}\psi(x,\>t)\approx \dfrac{1-\frac{1}{2}iH\Delta t}
    {1+\frac{1}{2}iH\Delta t}\psi(x,\>t)\>.
    \label{razvoj}
\end{equation}

Območje $x\in[a,\>b]$ diskretiziramo na krajevno mrežo z $N$ točkami $x_j = a + j\Delta x$, kjer je $\Delta x = (b-a)/(N-1)$.
Časovni razvoj spremljamo ob časovni mreži z $M$ točkami $t_m = m\Delta t$, kjer je $\Delta t$ časovni korak. Vrednosti valovne
funkcije in potenciala v mrežnih točkah ob času $t_m$ označimo z $\psi_j^m$ in $V_j$. Krajevni odvod izrazimo z 
diferenco

\begin{equation}
    \Psi''(x) \approx \frac{\psi(x + \Delta x,\>t)-2\psi(x,\>t)+\psi(x-\Delta x,\>t)}{\Delta x^2} 
    = \frac{\psi_{j+1}^m - 2\psi_j^m + \psi_{j-1}^m}{\Delta x^2}\>.
    \label{krajevni_odvod}
\end{equation}

Te približke vstavimo v razvoj stanja \ref{razvoj} in razpišemo Hamiltonov operator, da dobimo sistem enačb

\begin{multline}
    \psi_j^{m+1} - i \frac{\Delta t}{4\Delta x^2}\left[\psi_{j+1}^{m+1} - 2\psi_j^{m+1} + \psi_{j-1}^{m+1}\right] + i \frac{\Delta t}{2} V_j \psi_j^{m+1} = \\
    \psi_j^m + i \frac{\Delta t}{4\Delta x^2}\left[\psi_{j+1}^m - 2\psi_j^m + \psi_{j-1}^m\right] - i \frac{\Delta t}{2}V_j\psi_j^m\>.
    \label{sistem}
\end{multline}

v notranjih točkah mreže, medtem ko na robu (pri $j\le 0$ in $j\ge N$) postavimo $\psi_j^{m} = 0$. Vrednosti valovne funkcije uredimo v 
vektor $\Psi^m$ in sistem \ref{sistem} prepišemo v matrično obliko

\begin{equation}
    \mathbf{A}\Psi^{m+1} = \mathbf{A^*}\Psi^m\>,
\end{equation}

kjer je $\mathbf{A}$ tridiaonalna matrika z elementi

\begin{equation}
    b = 1 + i\frac{\Delta t}{2\Delta x^2}\>,\quad a = -\frac{b}{2}\>,\quad d_j = 1 + b + i\frac{\Delta t}{2}V_j\>.
\end{equation}
Torej je $\mathbf{A}$ oblike

\begin{equation}
    \mathbf{A} = \begin{bmatrix}
        d_1 & a & 0 & \cdots & 0 \\
        a & d_2 & a & \cdots & 0 \\
        0 & a & d_3 & \cdots & 0 \\
        \vdots & \vdots & \vdots & \ddots & \vdots \\
        0 & 0 & 0 & \cdots & d_{N-1}
    \end{bmatrix}\>.
\end{equation}

Kot smo napovedali, je to torej matrični sistem, ki ga moramo rešiti v vsakem časovnem koraku iterativno.

\section{Naloga}

\section{Opis reševanja}

\section{Rezultati}


\section{Komentarji in izboljšave}


\newpage
\bibliographystyle{unsrt}
\bibliography{sources}
\end{document}
